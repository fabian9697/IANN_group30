%%%%%%%%%%%%%%%%%%%%%%%%%%%%%%%%%%%%%%%%%
% Stylish Article
% LaTeX Template
% Version 2.2 (2020-10-22)
%
% This template has been downloaded from:
% http://www.LaTeXTemplates.com
%
% Original author:
% Mathias Legrand (legrand.mathias@gmail.com) 
% With extensive modifications by:
% Vel (vel@latextemplates.com)
% Jacqueline Näther (jnaether@uni-osnabrueck.com)
% Fabian Imkenberg (fimkenberg@uni-osnabrueck.com)
%
% License:
% CC BY-NC-SA 3.0 (http://creativecommons.org/licenses/by-nc-sa/3.0/)
%
%%%%%%%%%%%%%%%%%%%%%%%%%%%%%%%%%%%%%%%%%

%----------------------------------------------------------------------------------------
%	PACKAGES AND OTHER DOCUMENT CONFIGURATIONS
%----------------------------------------------------------------------------------------

\documentclass[fleqn,10pt]{SelfArx} % Document font size and equations flushed left

\usepackage[english]{babel} % Specify a different language here - english by default

\usepackage{lipsum} % Required to insert dummy text. To be removed otherwise

\usepackage[nolist]{acronym}

%----------------------------------------------------------------------------------------
%	COLUMNS
%----------------------------------------------------------------------------------------

\setlength{\columnsep}{0.55cm} % Distance between the two columns of text
\setlength{\fboxrule}{0.75pt} % Width of the border around the abstract

%----------------------------------------------------------------------------------------
%	COLORS
%----------------------------------------------------------------------------------------

\definecolor{color1}{RGB}{172,06,52} % Corporate Design Colour of University of Osnabrueck
\definecolor{color2}{RGB}{207,208,209} % Corporate Design Colour of University of Osnabrueck
\definecolor{color3}{RGB}{251,185,0} % Corporate Design Colour of University of Osnabrueck

%----------------------------------------------------------------------------------------
%	HYPERLINKS
%----------------------------------------------------------------------------------------

\usepackage{hyperref} % Required for hyperlinks

\hypersetup{
	hidelinks,
	colorlinks,
	breaklinks=true,
	urlcolor=color3,
	citecolor=color1,
	linkcolor=color1,
	bookmarksopen=false,
	pdftitle={Title},
	pdfauthor={Author},
}

%----------------------------------------------------------------------------------------
%	ARTICLE INFORMATION
%----------------------------------------------------------------------------------------

\JournalInfo{\includegraphics[width=0.3\linewidth]{logo_uni}}
\Archive{} %!!NEEDED, otherwise \tableofcontent error, is for Additional notes below university logo (e.g. copyright, DOI, review/research article)

\PaperTitle{Reimplementation of a Cycle-Consistent Adversarial Network for unpaired Image-to-Image Translation}

\Authors{Fabian Imkenberg, Jacqueline Naether}

\Keywords{CycleGAN --- Unpaired Image-to-Image Translation --- Keyword3} 
\newcommand{\keywordname}{Keywords} % Defines the keywords heading name

%----------------------------------------------------------------------------------------
%	ABSTRACT
%----------------------------------------------------------------------------------------

\Abstract{Lorem ipsum dolor sit amet, consectetuer adipiscing elit. Ut purus elit, vestibulum ut, placerat ac, adipiscing vitae, felis. Curabitur dictum gravida mauris. Nam arcu libero, nonummy eget, consectetuer id, vulputate a, magna. Donec vehicula augue eu neque. Pellentesque habitant morbi tristique senectus et netus et malesuada fames ac turpis egestas. Mauris ut leo. Cras viverra metus rhoncus sem. Nulla et lectus vestibulum urna fringilla ultrices. Phasellus eu tellus sit amet tortor gravida placerat. Integer sapien est, iaculis in, pretium quis, viverra ac, nunc. Praesent eget sem vel leo ultrices bibendum. Aenean faucibus. Morbi dolor nulla, malesuada eu, pulvinar at, mollis ac, nulla. Curabitur auctor semper nulla. Donec varius orci eget risus. Duis nibh mi, congue eu, accumsan eleifend, sagittis quis, diam. Duis eget orci sit amet orci dignissim rutrum.}

%----------------------------------------------------------------------------------------

\begin{document}
\maketitle
\tableofcontents
\thispagestyle{empty} % Removes page numbering from the first page

%----------------------------------------------------------------------------------------
%	ARTICLE CONTENTS
%----------------------------------------------------------------------------------------

\section*{Introduction}
\addcontentsline{toc}{section}{Introduction}
As the basis of our project, we studied the paper \textit{Unpaired Image-to-Image Translation using Cycle-Consistent Adversarial Networks} from the \ac{BAIR} laboratory. The aim of the paper is to change different information in images. The developed network creates a new generated image based on an existing input image. Three different types of generated images are presented in this paper. A translation of a photograph to a painting by Monet, zebras to horses, and summer to winter and vice versa. The paper complements what has been learned in the module \ac{IANNWTF} by two essential building blocks, which basically depend on each other.~\cite{image-to-image-ccan}

Until now, training data was mostly available in input-output pairs. This means that in the example translation summer to winter, for each photo of a location in summer there also exists a photo of the same location in winter. Thus, the network previously learned translation using complete individual examples. In the case of the paper, however, a network is developed that is trained based on two domains. As training input, the network is fed a domain of, for example, summer images. The training output then contains a domain of winter images. The network learns the different properties of summer and winter based on these image domains.~\cite{image-to-image-ccan}

As a second building block, the simple \ac{GAN} as presented in \ac{IANNWTF} is extended into a so-called Cycle-\ac{GAN}. The name is based on the use of the Cycle Consistency Loss in the training phase. We reimplemented and tested such a Cycle-\ac{GAN} based on the paper.~\cite{image-to-image-ccan}

%------------------------------------------------

\section{Methods}

As a transfer performance, our goal is not to teach our network the same translation as described in the paper.  Therefore, we decided to optimize our network to perform a translation from apple images to orange images and vice versa. For this, we had to select a suitable dataset, as well as to work out the methodological basics of the network architecture from the paper. In the following, these basics will be explained.

\subsection{Network Architecture}
As the basis of the network architecture, the \ac{GAN} already known through the \ac{IANNWTF} module is extended to the Cycle-\ac{GAN} known through the paper \cite{image-to-image-ccan}.

\paragraph{\acl{GAN}s} basically exist out of two neural networks, which more or less battle against each other, the \textit{Generator} and \textit{Discriminator}. The Generator takes the input and generates a fake image out of it, while the Discriminator tries to discriminate between generated and true images. Both parts of the network are trained individually from each other. The loss used in training is the adveresarial loss.~\cite{Introduction-to-Cycle-GANs, GAN-Courseware}

\paragraph{Cycle-\ac{GAN}s} extend this concept, in terms of unpaired image-to-image translation. As mentioned in the introduction, the choosen paper uses domains of input \textit{D\textsubscript{x}} and output images \textit{D\textsubscript{y}}. The Generator is trained two main functions. Generating images from domain \textit{D\textsubscript{x}} to images of domain \textit{D\textsubscript{y}} ($G: X \rightarrow Y$) and vice versa ($F: Y \rightarrow X$). This principle is shown in \autoref{fig:mappingFunctions}.~\cite{image-to-image-ccan}

\begin{figure} \centering 
% Using \begin{figure*} makes the figure take up the entire width of the page
	\includegraphics[width=0.8\linewidth]{mappingFunctions}
	\caption{Mapping functions of the Cycle-\ac{GAN}~\cite{image-to-image-ccan}}
	\label{fig:mappingFunctions}
\end{figure}

On the basis of this it becomes clear that a simple back and forward translation can lead to the fact that the back translation does not refer to exactly the original input image, but to another image within the input image domain. This is where the cycle consistency loss comes into play. This one is used in addition to the adversarial loss so that the network learns to return to the original image during a back and forward translation. The cycle consistency losses for the two mapping functions are shown in the \autoref{fig:cycleConsistencyLoss}.~\cite{image-to-image-ccan}

\begin{figure*}[htb] \centering 
% Using \begin{figure*} makes the figure take up the entire width of the page
	\includegraphics[width=\linewidth]{cycleConsistencyLoss}
	\caption{Cycle Consistency Losses in the back and forward translation~\cite{image-to-image-ccan}}
	\label{fig:cycleConsistencyLoss}
\end{figure*}


\subsection{Formulation}
In order to better understand the function and subsequent implementation of the losses, the mathematical background will be discussed in a bit more detail.

\paragraph{Adversarial Loss} is already used in training of standard \ac{GAN}s. It arises from the fact that the generator \textit{G} and discriminator \textit{D} work against each other. If the generator generates particularly good fake images that the discriminator can no longer identify as such, the loss of the discriminator automatically increases. On the other hand, the loss of the generator decreases. The loss behavior is similar the other way round. This is described mathematically in \cite{Source-GAN} as follows:
\begin{equation*}
\begin{split}
\min_{G} \max_{D} V(D,G) =&~\mathbb E_{x \sim p_{data}(x)} [\log D(x)] \\\
&+ \mathbb E_{z \sim p_{z}(z)} [\log (1-D(G(z)))]
\end{split}
\end{equation*}

\paragraph{Cycle Consistency Loss} is used additionally to the adversarial loss in the Cycle-\ac{GAN}. The idea has already been briefly explained before, that a generator \textit{G} translates an image \textit{Input\_ A} from the domain \textit{X} into an image \textit{B} of the domain \textit{Y}. In the backward translation of generator \textit{F}, an image \textit{Cyclic\_ A} from domain \textit{X} will be created from the image \textit{B} of domain \textit{Y}. The difference between image \textit{Input\_ A} and \textit{Cyclic\_ A} forms the cycle consistency loss \cite{Introduction-to-Cycle-GANs}. In \cite{Introduction-to-Cycle-GANs}, the cycle consistency loss is described mathematically as follows:
\begin{equation*}
	Loss_{cyc}(G,F,X,Y) = \frac{1}{m} \sum^{m}_{i=1}[F(G(x_i))-x_i]+[G(F(y_i))-y_i]
\end{equation*}

In summary, the Cycle-\ac{GAN} is built from two generators, \textit{Generator A2B} and \textit{Generator B2A}, and two discriminators, \textit{Discriminator A} und \textit{Discriminator B}. This architecture is shown simplified in \autoref{fig:simplified-cycle-gan}.

\begin{figure}[htb] 
	\centering 
	\includegraphics[width=\linewidth]{simplified-cycle-gan}
	\caption{A simplified architecture of a Cycle-\ac{GAN} \cite{Introduction-to-Cycle-GANs}}
	\label{fig:simplified-cycle-gan}
\end{figure}


\subsection{Dataset}
In addition to the architecture, a suitable dataset also had to be selected. On the one hand, it should be versatile, but on the other hand, it must not contain too large images so that the computing time is not too high.
\begin{enumerate}[noitemsep] % [noitemsep] removes whitespace between the items for a compact look
	\item describe which dataset we choose ?!
	\item what are the facts of the training/test data and images itself ?!
\end{enumerate} 



%------------------------------------------------
\section{Implementation}
The implementation is performed using the learned three steps, pre-processing, training and test from the module \ac{IANNWTF} \cite{implementingANsCourseware02, implementingANsCourseware03}. In addition, the methods mentioned in the section before are used to develop the Cycle-\ac{GAN} architecture. How this implementation is realized is now explained in more detailed.

\subsection{Architecture}
The architecture is built as explained in section methods before. A class \texttt{Cycle\_GAN\_Generator} and a class \texttt{Cycle\_ GAN\_Discriminator} are programmed. Two instances of each of the two classes are then created, the \texttt{generator\_A \_to\_B} and \texttt{generator\_B\_to\_A}, and the \texttt{discriminator \_A} and \texttt{discriminator\_B}.

\paragraph{Generator Class} inherits from the Keras Class \texttt{Layer}. The class in general is build up of three parts, the \textit{Encoder}, \textit{Transformer} and \textit{Decoder}. Each part consists again out of different layers. The structure of the generator is shown in \autoref{fig:generator}. \cite{Introduction-to-Cycle-GANs}

\begin{figure*}[htb] 
	\centering 
	\includegraphics[width=\linewidth]{generator}
	\caption{The high-level structure of the Cycle-\ac{GAN}s generator \cite{Introduction-to-Cycle-GANs}}
	\label{fig:generator}
\end{figure*}



\paragraph{Discriminator Class}

\subsection{Pre-Processing} 
\begin{enumerate}[noitemsep] % [noitemsep] removes whitespace between the items for a compact look
	\item Wie wurde der Dataset verarbeitet, Funktionen (shuffel, buffer, prefetch?)
	\item was haben wir für Werte genau ausgewählt und warum?
	\item Wie groß ist Training/Testdataset
\end{enumerate}

\subsection{Training}
\begin{enumerate}[noitemsep] % [noitemsep] removes whitespace between the items for a compact look
	\item Welche Schritte gibt es im Training? Generator/Discriminator
	\item Welche Klassen/Funktionen wurden hierfür dann geschrieben?
	\item Wo sind relevanten Werte gewählt worden? Warum wurden sie so gewählt?(z.B. Numbers of epoch?)
\end{enumerate}

\subsection{Test}
\begin{enumerate}[noitemsep] % [noitemsep] removes whitespace between the items for a compact look
	\item Welche Schritte gibt es im Test? Generator/Discriminator, einfacher Forward Step?
	\item Ausgabe/Ergebnis anzeigen, wie umgesetzt?
\end{enumerate}

%------------------------------------------------
\section{Evaluation}
\begin{enumerate}[noitemsep] % [noitemsep] removes whitespace between the items for a compact look
	\item Wir haben improvements anhand folgender zwei paper vorgenommen
\end{enumerate}

\subsection{Improved Paper 1}

\subsection{Improved Paper 2}

%\begin{table}[hbt]
%	\caption{Table of Grades}
%	\centering
%	\begin{tabular}{llr}
%		\toprule
%		\multicolumn{2}{c}{Name} \\
%		\cmidrule(r){1-2}
%		First name & Last Name & Grade \\
%		\midrule
%		John & Doe & $7.5$ \\
%		Richard & Miles & $2$ \\
%		\bottomrule
%	\end{tabular}
%	\label{tab:label}
%\end{table}

%------------------------------------------------

\section{Results}

\begin{enumerate}[noitemsep] % [noitemsep] removes whitespace between the items for a compact look
	\item Welche Ergebnisse erzielt das Training?
	\item Welches Ergebnis erzielt der Test?
\end{enumerate}


%------------------------------------------------

\section{Limitations and Discussion}

\begin{enumerate}[noitemsep] % [noitemsep] removes whitespace between the items for a compact look
	\item Datensatz/Bildgröße
	\item Dauer des Trainings und beanspruchung der rechenzeit
\end{enumerate}
%----------------------------------------------------------------------------------------
%	REFERENCE LIST
%----------------------------------------------------------------------------------------

\phantomsection
\bibliographystyle{unsrt}
\bibliography{sample}

%----------------------------------------------------------------------------------------

\begin{acronym}
\acro{BAIR}[BAIR]{Berkeley AI Research}
\acro{IANNWTF}[IANNWTF]{Implementing Artificial Networks with Tensorflow}
\acro{GAN}[GAN]{Generative Adversarial Network}
%\acroplural{Kuerzel}[Kurzform des Plurals]{Langform des Plurals}
\end{acronym}

%----------------------------------------------------------------------------------------


\end{document}